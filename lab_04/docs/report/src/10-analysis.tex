\chapter{Практическая часть}

    \section{Чем принципиально отличаются функции cons, list, append?}
    
        Пусть:
        \begin{lstlisting} [
            float=h!,
            frame=single,
            numbers=left,
            abovecaptionskip=-5pt,
            caption={объявление функций из условия},
            label={lst:1-1},
            language={Lisp},
        ]
        (setf lst1 '(a b))
        (setf lst2 '(c d))
        \end{lstlisting}
        
        В Таблице \ref{tbl:1-1} приведены результаты вычисления выражений.
        
        \begin{table}[!ht]
            \begin{center}
                \caption{Результаты вычисления выражений}
                \label{tbl:1-1}
                \begin{tabular}{|l|l|}
                    \hline
                    \bfseries Выражение & \bfseries Результат\\\hline
                    \texttt{(cons lst1 lst2)} & \texttt{((A B) C D)}\\\hline
                    \texttt{(list lst1 lst2)} & \texttt{((A B) (C D))}\\\hline
                    \texttt{(append lst1 lst2)} & \texttt{(A B C D)}\\\hline
                \end{tabular}
            \end{center}
        \end{table}
    
    \section{Каковы результаты вычисления следующих выражений?}
    
        В Таблице \ref{tbl:2-1} приведены результаты вычисления выражений.
        
        \begin{table}[!ht]
            \begin{center}
                \caption{Результаты вычисления выражений}
                \label{tbl:2-1}
                \begin{tabular}{|l|l|}
                    \hline
                    \bfseries Выражение & \bfseries Результат\\\hline
                    \texttt{(reverse ())} & \texttt{(Nil)}\\\hline
                    \texttt{(last ())} & \texttt{(Nil)}\\\hline
                    \texttt{(reverse '(a))} & \texttt{(a)}\\\hline
                    \texttt{(last '(a))} & \texttt{(a)}\\\hline
                    \texttt{(reverse '((a b c)))} & \texttt{((a b c))}\\\hline
                    \texttt{(last '((a b c)))} & \texttt{((a b c))}\\\hline
                \end{tabular}
            \end{center}
        \end{table}
    
    \section{Написать, по крайней мере, два варианта функции, которая возвращает последний элемент своего списка-аргумента}
    
        \begin{lstlisting} [
            float=h!,
            frame=single,
            numbers=left,
            abovecaptionskip=-5pt,
            caption={Вариант 1},
            label={lst:3-1},
            language={Lisp},
        ]
(defun mylast1 (lst)
  (cond ((null lst)Nil)
        ((null (cdr lst))lst)
        (T(mylast1 (cdr lst)))))
        \end{lstlisting}
        
        \begin{lstlisting} [
            float=h!,
            frame=single,
            numbers=left,
            abovecaptionskip=-5pt,
            caption={Вариант 2},
            label={lst:3-2},
            language={Lisp},
        ]
(defun mylast2 (lst)
  (let ((revlst (reverse lst)))
       (if (null lst)
           Nil
           (car revlst))))
        \end{lstlisting}
    
    \section{Написать, по крайней мере, два варианта функции, которая возвращает свой список-аргумент без последнего элемента}
    
        \begin{lstlisting} [
            float=h!,
            frame=single,
            numbers=left,
            abovecaptionskip=-5pt,
            caption={Вариант 1},
            label={lst:4-1},
            language={Lisp},
        ]
(defun removelast1 (lst)
  (cond ((null lst)Nil)
        ((null (cdr lst))Nil)
        (T(cons (car lst) (removelast (cdr lst))))))
        \end{lstlisting}
        
        \begin{lstlisting} [
            float=h!,
            frame=single,
            numbers=left,
            abovecaptionskip=-5pt,
            caption={Вариант 2},
            label={lst:4-2},
            language={Lisp},
        ]
(defun removelast2 (lst)
  (and lst (reverse (cdr (reverse lst)))))
        \end{lstlisting}
    
    \section{Написать простой вариант игры в кости, в котором бросаются две правильные кости}
    
    Если сумма выпавших очков равна 7 или 11 -- выигрыш, если выпало (1,1) или (6,6) --- игрок имеет право снова бросить кости, во всех остальных случаях ход переходит ко второму игроку, но запоминается сумма выпавших очков. Если второй игрок не выигрывает абсолютно, то выигрывает тот игрок, у которого больше очков. Результат игры и значения выпавших костей выводить на экран с помощью функции print.
    
        \begin{lstlisting} [
            float=h!,
            frame=single,
            numbers=left,
            abovecaptionskip=-5pt,
            caption={Игра в кости},
            label={lst:4-2},
            language={Lisp},
        ]
(defun is_step_win (cube1p cube2p)
  (let ((sum (+ cube1p cube2p)))
       (OR (= 7 sum)(= 11 sum))))

(defun is_step_repeat (cube1p cube2p)
  (AND (= cube1p cube2p) (OR (= 1 cube1p) (= 6 cube1p))))

(defun play_step (player_name)
  (let* ((cube1p (+ (random 8) 1))
         (cube2p (+ (random 8) 1))
         (sum (+ cube1p cube2p)))
        (cond ((is_step_win cube1p cube2p)
               (AND (print `(,player_name - points ,cube1p ,cube2p - ABSOLUTE_WIN))
                    (list T sum)))
              ((is_step_repeat cube1p cube2p)
               (AND (print `(,player_name - points ,cube1p ,cube2p - REPEAT))
                    (let ((res (play_step player_name)))
                         (list (first res)(+ sum (second res))))))
              (T(AND (print `(,player_name - points ,cube1p ,cube2p - TRANSFER))
                     (list Nil sum))))))

(defun play_game ()
  (let ((resu1 (play_step 'USER1)))
       (if (first resu1)
           (print `(USER1 is a winner))
           (let ((resu2 (play_step 'USER2)))
                (if (first resu2)
                    (print `(USER2 is a winner))
                    (cond ((> (second resu1)(second resu2))
                           (print '(USER1 is a winner)))
                          ((< (second resu1)(second resu2))
                           (print '(USER2 is a winner)))
                          (T(print '(DRAW)))))))))
        \end{lstlisting}
    