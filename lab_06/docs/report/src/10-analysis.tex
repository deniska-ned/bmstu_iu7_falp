\chapter{Практическая часть}

\section{Напишите функцию, которая уменьшает на 10 все числа из списка-аргумента этой функции}

\begin{lstlisting} [
	float=h!,
	frame=single,
	numbers=left,
	abovecaptionskip=-5pt,
	caption={Функция, проверяющая, является ли список палиндромом},
	label={lst:1-1},
	language={Lisp},
]
(defun minus-10 (lst) 
  (mapcar (lambda (x) (- x 10)) lst))
\end{lstlisting}

\section{Напишите функцию, которая умножает на заданное число-аргумент все числа из заданного списка-аргумента, когда\dots}

\subsection{а) все элементы списка -- числа}

\begin{lstlisting} [
	float=h!,
	frame=single,
	numbers=left,
	abovecaptionskip=-5pt,
	caption={Функция, умножающая каждый элемент списка на число},
	label={lst:2-1},
	language={Lisp},
]
(defun mult-all-numbers (mult lst)
  (mapcar #'(lambda (el) (* el mult)) lst))
\end{lstlisting}

\subsection{б) элементы списка -- любые объекты}

\begin{lstlisting} [
	float=h!,
	frame=single,
	numbers=left,
	abovecaptionskip=-5pt,
	caption={Функция, умножающая каждое число из списка на число},
	label={lst:2-2},
	language={Lisp},
]
(defun compl-mult-all-numbers (mult lst)
  (mapcar
    #'(lambda (el)
        (cond ((listp el) (compl-mult-all-numbers mult el))
              ((* el mult))))
    lst))
\end{lstlisting}

\section{Написать функцию, которая по своему списку-аргументу lst определяет является ли он палиндромом (то есть равны ли lst и (reverse lst))}

\begin{lstlisting} [
	float=h!,
	frame=single,
	numbers=left,
	abovecaptionskip=-5pt,
	caption={Функция, возводящая все элементы списка в квадрат},
	label={lst:3-1},
	language={Lisp},
]
(defun palindromp (lst)
  (every #'= lst (reverse lst)))
\end{lstlisting}

\section{Написать предикат set-equal, который возвращает t, если два его множества аргумента содержат одни и те же элементы, порядок которых не имеет значения.}

\begin{lstlisting} [
	float=h!,
	frame=single,
	numbers=left,
	abovecaptionskip=-5pt,
	caption={Функция, возводящая все элементы списка в квадрат},
	label={lst:5-1},
	language={Lisp},
]
(defun my-subsetp (s1 s2)
  (every
    #'(lambda (a) (some #'(lambda (b) (= a b)) s2))
    s1))
    
(defun set-equal (set1 set2)
  (cond
    ((/= (length set1) (length set2)) nil)
    ((my-subsetp set1 set2) (my-subsetp set2 set1))))
\end{lstlisting}

\section{Написать функцию которая получает как аргумент список чисел, а возвращает список
квадратов этих чисел в том же порядке.}

\begin{lstlisting} [
	float=h!,
	frame=single,
	numbers=left,
	abovecaptionskip=-5pt,
	caption={Функция, возводящая все элементы списка в квадрат},
	label={lst:5-1},
	language={Lisp},
]
(defun sqrlist (lst)
  (mapcar #'* lst lst))
\end{lstlisting}

\section{Напишите функцию, select-between, которая из списка-аргумента, содержащего только числа, выбирает только те, которые расположены между двумя указанными границами аргументами и возвращает их в виде списка}

\begin{lstlisting} [
	float=h!,
	frame=single,
	numbers=left,
	abovecaptionskip=-5pt,
	caption={Функция, выбирающая из списка-аргумента числа, расположенные между двумя указанными границами-аргументами},
	label={lst:6-1},
	language={Lisp},
]
(defun select-between (from to lst)
  (mapcan #'(lambda (el)
              ((< from el to) (list el)))
          lst))
\end{lstlisting}

\section{Написать функцию, вычисляющую декартово произведение двух своих списков аргументов. (Напомним, что А х В это множество всевозможных пар (a b), где а принадлежит А, а b принадлежит В)}

\begin{lstlisting} [
	float=h!,
	frame=single,
	numbers=left,
	abovecaptionskip=-5pt,
	caption={Функция, вычисляющая декартово произведение элементов списков аргументов},
	label={lst:7-1},
	language={Lisp},
]
(defun decart (lstA lstB)
    (mapcan #'(lambda (a)
                (mapcar #'(lambda (b)
                            (list x y)) lstB))
                                        lstA))
\end{lstlisting}

\section{Почему так реализовано reduce, в чем причина?}

\texttt{(reduce \#'+ ()) -> 0}

Поведение в данном примере обусловлено работой функции \texttt{+}. Эта функция --- функционал, который при 0 количестве аргументов возвращает значение 0. Если подать на вход \texttt{reduce} функцию, которая не может обработать 0 аргументов (например, математическая функция \texttt{cons}), то вызов \texttt{reduce} с пустым списком в качестве второго аргумента вернет ошибку (\texttt{invalid number of arguments: 0}). При этом, если подано более одного аргумента, то \texttt{reduce} выполняет следующие действия:
\begin{enumerate}
    \item сохраняет первый элемент списка в область памяти (для определенности назовем ее \texttt{acc});
    \item для всех остальных элементов списка выполняет переданную в качестве первого аргумента функцию, подавая на вход 2 аргумента (\texttt{acc} и очередной элемент списка) и сохраняя результат в \texttt{acc}.
\end{enumerate}

Пример упрощенной реализации \texttt{reduce} (в данной реализации опущены проверки аргументов):

\begin{lstlisting} [
	float=h!,
	frame=single,
	numbers=left,
	abovecaptionskip=-5pt,
	caption={Пример реализации reduce},
	label={lst:8-1},
	language={Lisp},
]
(defun my-reduce-internal (func lst acc)
  (cond ((null lst) acc)
        ((my-reduce-internal func (cdr lst) (funcall func acc (car lst))))))

(defun my-reduce (func lst)
  (cond ((null lst) (funcall func))
        ((my-reduce-internal func (cdr lst) (car lst)))))
\end{lstlisting}

\section{Пусть list-of-list список, состоящий из списков. Написать функцию, которая вычисляет сумму длин всех элементов list-of-list, т.е. например для аргумента ((1 2) (3 4)) -> 4}

\begin{lstlisting} [
	float=h!,
	frame=single,
	numbers=left,
	abovecaptionskip=-5pt,
	caption={Пример реализации reduce},
	label={lst:9-1},
	language={Lisp},
]
(defun sum-lens (list-of-lists)
  (reduce #'(lambda (acc lst) (+ acc (length lst)))
          (cons 0 list-of-lists)))
\end{lstlisting}