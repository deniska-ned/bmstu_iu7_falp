\chapter{Практическая часть}

\section{Написать функцию, которая принимает целое число и возвращает первое четное число, не меньшее аргумента}

\begin{lstlisting} [
	float=h!,
	frame=single,
	numbers=left,
	abovecaptionskip=-5pt,
	caption={Функция, возвращающая первое четное число, не меньшее аргумента},
	label={lst:1-1},
	language={Lisp},
]
(defun geeven (num) 
  (cond ((= 0 (mod num 2))num)
        (T(geeven (+ num 1)))))
\end{lstlisting}

\begin{lstlisting} [
	float=h!,
	frame=single,
	numbers=left,
	abovecaptionskip=-5pt,
	caption={Lambda-функция, возвращающая первое четное число, не меньшее аргумента},
	label={lst:1-1-2},
	language={Lisp},
]
(lambda (num) 
  (cond ((= 0 (mod num 2))num)
        (T(geeven (+ num 1)))))
\end{lstlisting}

\section{Написать функцию, которая принимает число и возвращает число того же знака, но с модулем на 1 больше модуля аргумента}

\begin{lstlisting} [
	float=h!,
	frame=single,
	numbers=left,
	abovecaptionskip=-5pt,
	caption={Функция, увеличивающая аргумент на 1 по модулю},
	label={lst:2-1},
	language={Lisp},
]
(defun incabs (num)
  (cond ((>= num 0)(+ num 1))
        (T(- num 1))))
\end{lstlisting}

\begin{lstlisting} [
	float=h!,
	frame=single,
	numbers=left,
	abovecaptionskip=-5pt,
	caption={Lmabda-функция, увеличивающая аргумент на 1 по модулю},
	label={lst:2-1-2},
	language={Lisp},
]
(lambda (num)
  (cond ((>= num 0)(+ num 1))
        (T(- num 1))))
\end{lstlisting}

\section{Написать функцию, которая принимает два числа и возвращает список из этих чисел, расположенный по возрастанию}

\begin{lstlisting} [
	float=h!,
	frame=single,
	numbers=left,
	abovecaptionskip=-5pt,
	caption={Функция, упорядочивающая аргументы по возрастанию},
	label={lst:3-1},
	language={Lisp},
]
(defun sort2 (num1 num2)
  (cond ((>= num2 num1)(list num1 num2))
        (T(list num2 num1))))
\end{lstlisting}

\begin{lstlisting} [
	float=h!,
	frame=single,
	numbers=left,
	abovecaptionskip=-5pt,
	caption={Lambda-функция, упорядочивающая аргументы по возрастанию},
	label={lst:3-1-2},
	language={Lisp},
]
(lambda (num1 num2)
  (cond ((>= num2 num1)(list num1 num2))
        (T(list num2 num1))))
\end{lstlisting}

\section{Написать функцию, которая принимает три числа и возвращает Т только тогда, когда первое число расположено между вторым и третьим}

\begin{lstlisting} [
	float=h!,
	frame=single,
	numbers=left,
	abovecaptionskip=-5pt,
	caption={Функция проверки принадлежности интервалу},
	label={lst:4-1},
	language={Lisp},
]
(defun inmed(num1 num2 num3)
  (cond ((AND (<= num2 num1)(<= num1 num3))T)
        (T Nil)))
\end{lstlisting}

\begin{lstlisting} [
	float=h!,
	frame=single,
	numbers=left,
	abovecaptionskip=-5pt,
	caption={Lamba-функция проверки принадлежности интервалу},
	label={lst:4-1-2},
	language={Lisp},
]
(lambda (num1 num2 num3)
  (cond ((AND (<= num2 num1)(<= num1 num3))T)
        (T Nil)))
\end{lstlisting}

\section{Каков результат вычисления следующих выражений}

В Таблице \ref{tbl:5-1} приведены результаты вычисления выражений, а так же варианты устранения возникших ошибок.

\begin{table}[!ht]
	\begin{center}
		\caption{Результаты вычисления выражений}
		\label{tbl:5-1}
		\begin{tabular}{|l|l|}
			\hline
			\bfseries Выражение & \bfseries Результат\\\hline
            \texttt{(and 'fee 'fie 'foe)} & \texttt{foe}\\\hline
            \texttt{(or 'fee 'fie 'foe)} & \texttt{fee}\\\hline
            \texttt{(or nil 'fie 'foe)} & \texttt{fie}\\\hline
            \texttt{(and nil 'fie 'foe)} & \texttt{nil}\\\hline
            \texttt{(and (equal 'abc 'abc) 'yes)} & \texttt{yes}\\\hline
            \texttt{(or (equal 'abc 'abc) 'yes)} & \texttt{T}\\\hline
		\end{tabular}
	\end{center}
\end{table}


\section{Написать предикат, который принимает два числа-аргумента и возвращает Т, если первое число не меньше второго}

\begin{lstlisting} [
	float=h!,
	frame=single,
	numbers=left,
	abovecaptionskip=-5pt,
	caption={Предикат сравнения чисел},
	label={lst:6-1},
	language={Lisp},
]
(defun predicate-2 (a b) (>= a b))
\end{lstlisting}

\section{Какой из следующих двух вариантов предиката ошибочен и почему}

\begin{lstlisting} [
	float=h!,
	frame=single,
	numbers=left,
	abovecaptionskip=-5pt,
	caption={Первая реализация предиката},
	label={lst:7-1},
	language={Lisp},
]
(defun pred1 (x) (and (numberp x) (plusp x))) 
\end{lstlisting}

\begin{lstlisting} [
	float=h!,
	frame=single,
	numbers=left,
	abovecaptionskip=-5pt,
	caption={Вторая реализация предиката},
	label={lst:7-2},
	language={Lisp},
]
(defun pred2 (x) (and (plusp x) (numberp x)))
\end{lstlisting}

\textbf{Ошибочен второй вариант}, потому что функция \texttt{plusp} принимает на вход один аргумент типа \texttt{number} и проверять, является ли аргумент числом, после выполнения функции \texttt{plusp} не имеет смысла, причем аргументы, не являющиеся числами, будут вызывать ошибку, в то время как 1 вариант будет работать с любым аргументом и возвращать \texttt{T} для положительных чисел.


\section{Решить задачу 4, используя для ее решения конструкции IF, COND, AND/OR}

Используя \texttt{if}:

\begin{lstlisting} [
	float=h!,
	frame=single,
	numbers=left,
	abovecaptionskip=-5pt,
	caption={Реализация с использованием \texttt{if}},
	label={lst:8-1},
	language={Lisp},
]
(defun inmed(num1 num2 num3)
  (if (AND (<= num2 num1)(<= num1 num3))
      T
      NIL))
\end{lstlisting}

Используя \texttt{cond}:

\begin{lstlisting} [
	float=h!,
	frame=single,
	numbers=left,
	abovecaptionskip=-5pt,
	caption={Реализация с использованием \texttt{cond}},
	label={lst:8-2},
	language={Lisp},
]
(defun inmed(num1 num2 num3)
  (cond ((AND (<= num2 num1)(<= num1 num3))T)
        (T Nil)))
)
\end{lstlisting}

Используя \texttt{and/or}:

\begin{lstlisting} [
	float=h!,
	frame=single,
	numbers=left,
	abovecaptionskip=-5pt,
	caption={Реализация с использованием \texttt{and/or}},
	label={lst:8-3},
	language={Lisp},
]
(defun inmed(num1 num2 num3)
  (AND (<= num2 num1)(<= num1 num3)))
\end{lstlisting}

\section{Переписать функцию how-alike, приведенную в лекции и использующую COND, используя
только конструкции IF, AND/OR}

Используя \texttt{cond}:

\begin{lstlisting} [
	float=h!,
	frame=single,
	numbers=left,
	abovecaptionskip=-5pt,
	caption={Реализация с использованием \texttt{cond}},
	label={lst:9-1},
	language={Lisp},
]
(defun how-alike (x y)
  (cond ((or(= x y) (equal x y)) 'the_same)
        ((and (oddp x) (oddp y)) 'both_odd) 
        ((and (evenp x) (evenp y)) 'both_even) 
        (T 'diff)))
\end{lstlisting}

Используя \texttt{if}:

\begin{lstlisting} [
	float=h!,
	frame=single,
	numbers=left,
	abovecaptionskip=-5pt,
	caption={Реализация с использованием \texttt{if}},
	label={lst:9-2},
	language={Lisp},
]
(defun how-alike-if (x y)
  (if (or (= x y) (equal x y)) 'the_same
    (if (and (oddp x) (oddp y)) 'both_odd
      (if (and (evenp x) (evenp y)) 'both_even
        'diff))))
\end{lstlisting}

Используя \texttt{and/or}:

\begin{lstlisting} [
	float=h!,
	frame=single,
	numbers=left,
	abovecaptionskip=-5pt,
	caption={Реализация с использованием \texttt{if}},
	label={lst:9-3},
	language={Lisp},
]
(defun how-alike-and-or (x y)
  (or (and (or (= x y) (equal x y)) 'the_same)
        (and (and (oddp x) (oddp y)) 'both_odd)
        (and (and (evenp x) (evenp y)) 'both_even) 
        'diff))
\end{lstlisting}